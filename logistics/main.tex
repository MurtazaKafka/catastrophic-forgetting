 
\documentclass[12pt]{article}

\usepackage[margin=1in]{geometry}
\usepackage{setspace, url}

\begin{document}

\setstretch{1.25}
\renewcommand*\arraystretch{1.5}
\pagenumbering{gobble}

\begin{center}
  {\Large \textbf{CSC 495 -- Computational Neuroethology: Video-Based Behavioral Analysis}} \\
  ~\\
  Davidson College \\
  Fall 2023 \\
\end{center}

\subsubsection*{Instructor}
Prof. Raghuram Ramanujan \\
\url{raramanujan@davidson.edu}

\subsubsection*{Student}
Shahin Ahmadi \\
\url{shahmadi@davidson.edu}

\subsubsection*{Class Meeting Times}
TBD

\subsection*{About the Course}
Computational neuroethology is a booming interdisciplinary field that combines principles from computer science, neuroscience, and ethology to understand and analyze animal behavior. In this self-guided course, the student explores computational neuroethology and gains practical skills in machine learning, temporal modeling, transfer learning, video classification, and action recognition. By the course's conclusion, the student will develop a model capable of autonomously extracting meaningful behavioral information from real-world video data of animals.

\subsection*{Access Statement}

The college welcomes requests for accommodations related to disability and will grant those that are determined to be reasonable and maintain the integrity of a program or curriculum. To make such a request or to begin a conversation about a possible request, please contact the Office of Academic Access and Disability Resources by emailing \url{AADR@davidson.edu}.   It is best to submit accommodation requests within the drop/add period; however, requests can be made at any time in the semester.  Please keep in mind that accommodations are not retroactive.


\subsection*{Learning Outcomes}

The overall goal of the course is to enable the student to pursue an extended
independent research project, with the ultimate goal of publishing the results
in a professional venue. The primary learning outcomes in this course are
centered around the acquisition of effective research skills. Specifically, by
the end of the course, the student will be able to:

\begin{itemize}
\item effectively search the relevant computer science literature, and
  read and understand technical papers at an appropriate level,
\item correctly design computational experiments, with a view to
  answering questions such as ``Is algorithm A better than algorithm B
  for this problem?'',
\item design and implement algorithms, collect data, determine
  appropriate visualizations and analyze the results with basic
  statistical techniques, and
\item deliver a presentation for a mixed audience of specialists
  and non-specialists that summarizes his results in an engaging and
  effective manner.
\end{itemize}


\subsection*{Course Expectations and Assessment}

Since this course is research focused, the assessment will primarily be based on
the quality of the student's research efforts, rather than the results. The
student will be expected to meet with the instructor at least twice a week to
discuss work in progress and ideas for future work ($50\%$). The student will be
expected to keep an up to date logbook and make regular commits to a GitHub
repository viewable by the instructor, with appropriate code documentation
following PEP-8 standards ($25\%)$. The student will be expected to summarize
and present their results in a presentation at a Math/Computer Science Coffee
event to an audience comprising faculty and students from across the College.
This presentation will contribute $25\%$ towards the student's final grade.
\emph{The expected time commitment, including meetings, is 12 hours per week.
The student will complete all required work for the class by Friday, December
15, 2023.} 


\end{document}
